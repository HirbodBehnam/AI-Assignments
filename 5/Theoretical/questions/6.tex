\section*{سوال 6}
\begin{enumerate}
    \item تعداد متغیر‌های وابسته به هر یک از متغیر‌ها را ذکر می‌کنیم.
    \begin{itemize}
        \item $A$: فقط به $B$ وابسته است پس جدول $2^2$ حالت دارد.
        \item $B$: به چیزی وابسته نیست پس جدول $2$ حالت دارد.
        \item $C$: فقط به $B$ وابسته است پس جدول $2^2$ حالت دارد.
        \item $D$: فقط به $B$ وابسته است پس جدول $2^2$ حالت دارد.
        \item $E$: به $B$ و $C$ وابسته است پس جدول $2^3$ حالت دارد.
        \item $F$: به $E$ و $C$ و $D$ وابسته است پس جدول $2^4$ حالت دارد.
    \end{itemize}
    \item نه لزوما. ممکن است که یال‌های اضافی نیز رسم شده باشد. با این کار صرفا دیتایی که قرار است نگه داریم را زیاد می‌کنیم.
    همچنین حتی می‌توانیم جهت یال‌ها را نیز عوض کنیم. با این کار شهود مسئله زیر سوال می‌رود ولی همچنان به جواب درست می‌رسیم.
    برای نمونه می‌توانید اسلاید جلسه‌ی 14 صفحه‌ی 34 را ببینید که چه طور صرفا
    \lr{reorder}
    کردن می‌تواند یال‌های جدیدی درست کند.
    \item کافی است که گراف کاملی کنیم شبکه‌ی بالا را. مثلا یالی از $D$ به $A$ رسم کنیم.
    \item برای گراف کامل لزوما معنای خاصی نمی‌تواند داشته باشد. همان طور که در قسمت ب گفتم وجود یال نشان دهنده‌ی وابسته بودن قطعی نیست.
    اما اگر از طرفی گراف تهی باشد نشان‌دهنده‌ی این است که تمامی متغیر‌های ما مستقل از هم هستند.
\end{enumerate}