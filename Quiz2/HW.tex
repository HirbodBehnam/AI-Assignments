% !TEX program = xelatex
\documentclass[]{article}
\usepackage{commons/course}

\begin{document}
\printheader

\smalltitle{سوال 1}
\\\noindent
در ابتدا تمام دامنه متغیر‌ها را می‌نویسیم:
\begin{gather*}
    A = \{0, 1, 2, 3, 4, 5, 6, 7, 8, 9\}\\
    B = \{0, 1, 2, 3, 4, 5, 6, 7, 8, 9\}\\
    C = \{0, 1, 2, 3, 4, 5, 6, 7, 8, 9\}
\end{gather*}
سپس یال
$A \rightleftharpoons B$
را بررسی می‌کنیم. در این یال مشخص است که
$B$
نمی‌تواند صفر باشد یا
$A$
برابر 9 باشد چرا که دامنه‌ی آن یکی متغیر به مشکل می‌خورد. پس داریم:
\begin{gather*}
    A = \{0, 1, 2, 3, 4, 5, 6, 7, 8, \cancel{9}\}\\
    B = \{\cancel{0}, 1, 2, 3, 4, 5, 6, 7, 8, 9\}\\
    C = \{0, 1, 2, 3, 4, 5, 6, 7, 8, 9\}
\end{gather*}
حال یال
$B \rightleftharpoons C$
را بررسی می‌کنیم.
از طرفی به خاطر دامنه‌ی
$C$
مقادیر
$B$
باید حتما کوچکتر مساوی 3 باشد. همچنین از سمت دیگر مقادیر
$C$
باید مربع کامل باشند. همچنین دقت کنید که قبلا 0 را از دامنه‌ی $B$ خط زده بودیم. پس در نهایت داریم:
\begin{gather*}
    A = \{0, 1, 2, 3, 4, 5, 6, 7, 8, \cancel{9}\}\\
    B = \{\cancel{0}, 1, 2, 3, \cancel{4, 5, 6, 7, 8, 9}\}\\
    C = \{\cancel{0}, 1, \cancel{2, 3}, 4, \cancel{5, 6, 7, 8}, 9\}
\end{gather*}
در نهایت چون به دامنه‌ی
$B$
دست زدیم، دوباره دامنه‌ی
$A$
را نیز بررسی می‌کنیم.
\begin{gather*}
    A = \{0, 1, 2, \cancel{3, 4, 5, 6, 7, 8}, \cancel{9}\}\\
    B = \{\cancel{0}, 1, 2, 3, \cancel{4, 5, 6, 7, 8, 9}\}\\
    C = \{\cancel{0}, 1, \cancel{2, 3}, 4, \cancel{5, 6, 7, 8}, 9\}
\end{gather*}

\smalltitle{سوال 2}
\begin{enumerate}
    \item \phantom{kir} \begin{figure*}[h]
        \centering
        \begin{forest}
        [3
            [2
                [6]
                [2]
            ]
            [0
                [0]
                [10]
            ]
            [3
                [3]
                [5]
                [12]
            ]
        ]
        \end{forest}
    \end{figure*}
    \item مقادیر $\alpha$ و $\beta$ از چپ به راست در هر راس نوشته شده است.
    \begin{figure*}[h]
        \centering
        \begin{forest}
        [{$2, \infty$}
            [{$-\infty, 2$}
                [6]
                [2]
            ]
            [{$2, 0$}
                [0]
                [10,edge={crossedoutedge}]
            ]
            [{$2, 3$}
                [3]
                [5]
                [12]
            ]
        ]
        \end{forest}
    \end{figure*}
    پس بله هرس رخ می‌دهد.
    \item برای هر حالت min امیدریاضی را حساب می‌کنیم:
    \begin{center}
    \begin{forest}
        [{$4.83$}
            [{$0.5 \times 2 + 0.5 (\frac{6 + 2}{2}) = 3$}
                [6]
                [2]
            ]
            [{$0.5 \times 0 + 0.5 (\frac{0 + 10}{2}) = 2.5$}
                [0]
                [10]
            ]
            [{$0.5 \times 3 + 0.5 (\frac{3 + 5 + 12}{3}) = 4.83$}
                [3]
                [5]
                [12]
            ]
        ]
    \end{forest}
    \end{center}
    از هم بهتر است که یال سوم را انتخاب بکنیم.
\end{enumerate}

\end{document}
