\smalltitle{سوال 6}
از
\lr{Weltch's t-test}
استفاده می‌کنیم.
\begin{gather*}
    t = \frac{\bar{X_1} - \bar{X_2}}{\sqrt{\frac{\sigma^2_1}{N_1}+\frac{\sigma^2_2}{N_2}}}\\
    X_1 = \frac{60 + 42 + 41 + 39 + 38}{5} = 44 \quad X_2 = \frac{69 + 68 + 64 + 62 + 56 + 42 + 38}{7} = 57\\
    \sigma_1^2 = 82.5 \quad \sigma_2^2 = 154.3\\
    t = \frac{57 - 44}{\sqrt{\frac{82.5}{5} +\frac{154.3}{7}}} = 2.094\\
    df = 7 + 5 - 2 = 10
\end{gather*}
حال
\lr{p value}
را حساب می‌کنیم. این کار را به کمک زبان
\lr{R}
و
\lr{package p-test}
انجام می‌دهیم.
\begin{latin}
    \[ %https://stats.stackexchange.com/a/45156/359756
        p-value = \texttt{2*pt(2.094, df = 10, lower=FALSE)} = 0.064
    \]
\end{latin}
اگر
$\alpha = 0.05$
باشد، از آنجا که
$0.064 > 0.05$
است پس می‌توان نتیجه گرفت که موثر است.



