\section*{سوال 2}
فرض می‌کنیم که
\lr{activation threshold}
برابر 0 است. یعنی برای ورودی‌هایی که جمع وزن‌دار آنها بیشتر مساوی 0 شود خروجی 1 می‌شود. همچنین دقت کنید که
$\text{XOR}(a, b) = \text{OR}(\text{NOR}(a, b), \text{AND}(a, b))$
است. پس در لایه‌ی اول باید
\lr{NOR} و \lr{AND}
را طراحی کنیم. برای طراحی
\lr{AND}
کافی است که دو ورودی را با هم جمع کنیم و وقتی خروجی را 1 بدیم که جمع دو ورودی بیشتر از 1 باشد. یعنی عملا
برای
\lr{AND}
به صورت زیر است شبکه عصبی ما:
\begin{latin}
    \centering
    \begin{tikzpicture}[->, >=stealth', auto, semithick, node distance=2cm]
    \tikzstyle{every state}=[fill=white,draw=black,thick,text=black,scale=1]
    \node[state] (main)[] {};
    \node[] (in1)[above left of=main] {$a$};
    \node[] (in2)[below left of=main] {$b$};
    \node[] (const)[left of=main] {$1$};
    \path
    (in1) edge[above] node{$1$} (main)
    (in2) edge[below] node{$1$} (main)
    (const) edge[above] node{$-2$} (main)
    ;
    \end{tikzpicture}
\end{latin}
برای
\lr{NOR}
به صورت مشابه عمل می‌کنیم.
\begin{latin}
    \centering
    \begin{tikzpicture}[->, >=stealth', auto, semithick, node distance=2cm]
    \tikzstyle{every state}=[fill=white,draw=black,thick,text=black,scale=1]
    \node[state] (main)[] {};
    \node[] (in1)[above left of=main] {$a$};
    \node[] (in2)[below left of=main] {$b$};
    \node[] (const)[left of=main] {$1$};
    \path
    (in1) edge[above] node{$-1$} (main)
    (in2) edge[below] node{$-1$} (main)
    (const) edge[above] node{$2$} (main)
    ;
    \end{tikzpicture}
\end{latin}
برای خود
\lr{OR}
نیز داریم:
\begin{latin}
    \centering
    \begin{tikzpicture}[->, >=stealth', auto, semithick, node distance=2cm]
    \tikzstyle{every state}=[fill=white,draw=black,thick,text=black,scale=1]
    \node[state] (main)[] {};
    \node[] (in1)[above left of=main] {$a$};
    \node[] (in2)[below left of=main] {$b$};
    \node[] (const)[left of=main] {$1$};
    \path
    (in1) edge[above] node{$1$} (main)
    (in2) edge[below] node{$1$} (main)
    (const) edge[above] node{$1$} (main)
    ;
    \end{tikzpicture}
\end{latin}
پس در کل شبکه عصبی ما برابر است با:
\begin{latin}
    \centering
    \begin{tikzpicture}[->, >=stealth', auto, semithick, node distance=2cm]
    \node[state] (main_3) {};
    \node[state] (main_1)[above left = of main_3] {};
    \node[] (in1_1)[above left of=main_1] {$a$};
    \node[] (in2_1)[below left of=main_1] {$b$};
    \node[] (const_1)[left of=main_1] {$1$};
    \node[state] (main_2)[below left = of main_3] {};
    \node[] (in1_2)[above left of=main_2] {$a$};
    \node[] (in2_2)[below left of=main_2] {$b$};
    \node[] (const_2)[left of=main_2] {$1$};
    \node[] (const_3)[left of=main_3] {$1$};
    \node[] (xor)[right of=main_3] {XOR};
    \path
    (in1_1) edge[above] node{$1$} (main_1)
    (in2_1) edge[below] node{$1$} (main_1)
    (const_1) edge[above] node{$-2$} (main_1)
    (in1_2) edge[above] node{$-1$} (main_2)
    (in2_2) edge[below] node{$-1$} (main_2)
    (const_2) edge[above] node{$2$} (main_2)
    (main_2) edge[below] node{$1$} (main_3)
    (main_1) edge[above] node{$1$} (main_3)
    (const_3) edge[above] node{$1$} (main_3)
    (main_3) edge[above] node{} (xor)
    ;
    \end{tikzpicture}
\end{latin}