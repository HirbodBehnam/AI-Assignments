% !TEX program = xelatex
\documentclass[]{article}
\usepackage{commons/course}

\begin{document}
\printheader

\smalltitle{سوال 1}
\\\noindent
\begin{latin}
    \centering
    \begin{tikzpicture}[->, >=stealth', auto, semithick, node distance=2cm]
    \tikzstyle{every state}=[fill=white,draw=black,thick,text=black,scale=1]
    \node[state] (M)  {$M$};
    \node[state] (H)[below right of=M] {$H$};
    \node[state] (S)[above right of=H] {$S$};
    \path
    (M) edge[below] node{} (H)
    (S) edge[below] node{} (H)
    ;
    \end{tikzpicture}
\end{latin}
\vspace{1cm}
\begin{latin}
    \centering
    \begin{tabular}{|c|c|}
        \hline
        M & P(M)\\
        \hline
        +m & 0.1\\
        \hline
        -m & 0.9\\
        \hline
    \end{tabular}
    \quad
    \begin{tabular}{|c|c|}
        \hline
        S & P(S)\\
        \hline
        +s & 0.2\\
        \hline
        -s & 0.8\\
        \hline
    \end{tabular}
    \quad
    \begin{tabular}{|c|c|c|c|}
        \hline
        M & S & H & P(H | M, S)\\
        \hline
        +m & +s & +h & 0.9\\
        \hline
        +m & +s & -h & 0.1\\
        \hline
        +m & -s & +h & 0.5\\
        \hline
        +m & -s & -h & 0.5\\
        \hline
        -m & +s & +h & 0.6\\
        \hline
        -m & +s & -h & 0.4\\
        \hline
        -m & -s & +h & 0.1\\
        \hline
        -m & -s & -h & 0.9\\
        \hline
    \end{tabular}
\end{latin}
\smalltitle{سوال 2}
\\\noindent
در این قسمت باید حساب کنیم که
$P(S, M | H) = P(S | H) \times P(M | H)$
هست یا خیر. بدین منظور با توجه به باز شده‌ی فرمول‌های فوق قسمت چپ و راست را حساب می‌کنیم:
\begin{gather*}
    P(S, M | H) = \frac{P(S, M, H)}{P(H)} = \frac{P(S) \times P(M | S) \times P(H | S, M)}{0.9 \times 0.1 \times 0.2 + 0.5 \times 0.1 \times 0.8 + 0.6 \times 0.9 \times 0.2 + 0.1 \times 0.9 \times 0.8}\\
    = \frac{0.2 \times 0.1 \times 0.9}{0.238} = \frac{0.018}{0.238} \approx 0.076
\end{gather*}
از طرفی دیگر برای قسمت دوم تساوی داریم:
\begin{gather*}
    P(S | H) = \frac{P(S, H)}{P(H)} = \frac{P(S) \times P(H | S)}{P(H)} = \frac{0.2 \times (0.9 \times 0.1 + 0.6 \times 0.9)}{0.238} \approx 0.53\\
    P(M | H) = \frac{P(M, H)}{P(H)} = \frac{P(M) \times P(H | M)}{P(H)} = \frac{0.1 \times (0.9 \times 0.2 + 0.5 \times 0.8)}{0.238} \approx 0.24\\
    P(S | H) \times P(M | H) = 0.1272
\end{gather*}
پس این دو عدد برابر نیستند و مستقل نیستند.


\smalltitle{سوال 3}
\\\noindent
\begin{latin}
    \centering
    \begin{tikzpicture}[->, >=stealth', auto, semithick, node distance=3cm]
    \tikzstyle{every state}=[fill=white,draw=black,thick,text=black,scale=1]
    \node[state] (H) {$H$};
    \node[state] (M)[right of=H] {$M$};
    \node[state] (S)[right of=M] {$S$};
    \path
    (H) edge[below] node{} (M)
    (M) edge[below] node{} (S)
    ;
    \end{tikzpicture}
\end{latin}
برای رسم
\lr{CPT}ها
از جدول قسمت اول استفاده می‌کنیم. به عنوان مثال برای بدست آوردن جدول
\lr{H}ها
از جمع احتمال‌ها بر روی
\lr{H+}ها
استفاده می‌کنیم ضرب در احتمال آمدن هر کدام از افراد.
\begin{gather*}
    H^+ = 0.9 \times 0.1 \times 0.2 + 0.5 \times 0.1 \times 0.8 + 0.6 \times 0.9 \times 0.2 + 0.1 \times 0.9 \times 0.8 = 0.238\\
    H^- = 0.1 \times 0.1 \times 0.2 + 0.5 \times 0.1 \times 0.8 + 0.4 \times 0.9 \times 0.2 + 0.9 \times 0.9 \times 0.8 = 0.762
\end{gather*}
حال باید
$P(M | H)$
را بدست بیاریم که در قسمت قبل بدست آوردیم که برابر
\lr{0.24}
است. حال
$P(S | M)$
را بدست میاریم. این عبارت برابر است با
$\frac{P(S, M)}{P(M)}$
که از آنجا که
$M$ و $S$
مستقل هستند عبارت برابر می‌شود با
$\frac{P(S)P(M)}{P(M)} = P(S)$


\smalltitle{سوال 4}
\\\noindent
خیر همچنان مستقل هستند. چرا که صرفا این مورد علت تمدید فقط است.
\end{document}
