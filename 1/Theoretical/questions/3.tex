\smalltitle{سوال 3}
\link{https://s3-us-west-2.amazonaws.com/cs188websitecontent/exams/su15\_midterm1\_solutions.pdf}{منبع}
\begin{enumerate}
    \item بله. چرا که اصلا در صورتی که $h$ \lr{admissible} باشد می‌دانیم که $A^*$ بهینه است. و می‌دانیم که \lr{DFS} بهینه نیست.
    یعنی ممکن است که \lr{DFS} نود‌های بیشتری از درخت جست و جو را باز کند.
    \item کمتر یا مساوی! در صورتی که $h = 0$ قرار دهیم الگوریتم $A^*$ مانند \lr{UCS} عمل می‌کند.
    \item غلط است. ممکن است که دو برابر کردن تابع $h$ کاری کنیم که در یکی از حالات این تابع بدبینانه بشود.
    \item بله. دقت کنید که در صورتی که $h$ \lr{admissible} باشد باید حداکثر به اندازه‌ی جواب بهینه خروجی دهد. پس $h_2$ دو برابر جواب بهینه ممکن است که خروجی دهد.
    \item درست است. چرا که با میانگین گیری صرفا $h$ هر کدام از حالات را کم می‌کنیم. پس همچنان
    \lr{admissible}
    می‌ماند.
\end{enumerate}