\smalltitle{سوال 1}
\begin{enumerate}
    \item \phantom{kir} \\
    \begin{latin}
    \begin{center}
        \begin{tabular}{|l|l|}
            \hline
            Step & Fringe\\
            \hline
            1 & S \\
            \hline
            2 & \cancel{S} (S, A) (S, G) \\
            \hline
            3 & \cancel{S} \cancel{(S, A)} (S, G) (S, A, B) (S, A, C) \\
            \hline
            4 & \cancel{S} \cancel{(S, A)} \cancel{(S, G)} (S, A, B) (S, A, C) \\
            \hline
        \end{tabular}
    \end{center}
    \end{latin}
    \item \phantom{kir} \\
    \begin{latin}
    \begin{center}
        \begin{tabular}{|l|l|}
            \hline
            Step & Fringe\\
            \hline
            1 & S \\
            \hline
            2 & \cancel{S} (S, A) (S, G) \\
            \hline
            3 & \cancel{S} \cancel{(S, A)} (S, A, B) (S, A, C) (S, G) \\
            \hline
            4 & \cancel{S} \cancel{(S, A)} \cancel{(S, A, B)} (S, A, B, D) (S, A, C) (S, G) \\
            \hline
            5 & \cancel{S} \cancel{(S, A)} \cancel{(S, A, B)} \cancel{(S, A, B, D)} (S, A, B, D, E) (S, A, C) (S, G) \\
            \hline
            6 & \cancel{S} \cancel{(S, A)} \cancel{(S, A, B)} \cancel{(S, A, B, D)} \cancel{(S, A, B, D, E)} (S, A, B, D, E, G) (S, A, C) (S, G) \\
            \hline
            7 & \cancel{S} \cancel{(S, A)} \cancel{(S, A, B)} \cancel{(S, A, B, D)} \cancel{(S, A, B, D, E)} \cancel{(S, A, B, D, E, G)} (S, A, C) (S, G) \\
            \hline
        \end{tabular}
    \end{center}
    \end{latin}
    \item \phantom{kir} \\
    \begin{latin}
    \begin{center}
        \begin{tabular}{|l|l|}
            \hline
            Step & Fringe\\
            \hline
            1 & \begin{tabular}{@{}c@{}}(S, 0)\end{tabular}  \\
            \hline
            2 & \begin{tabular}{@{}c@{}} \cancel{(S, 0)} (S, A, 3) (S, G, 16) \end{tabular} \\
            \hline
            3 & \begin{tabular}{@{}c@{}} \cancel{(S, 0)} \cancel{(S, A, 3)} (S, A, C, 5) (S, A, B, 7) (S, G, 16) \end{tabular} \\
            \hline
            4 & \begin{tabular}{@{}c@{}} \cancel{(S, 0)} \cancel{(S, A, 3)} \cancel{(S, A, C, 5)} (S, A, B, 7) (S, A, C, 7) (S, G, 16) \end{tabular} \\
            \hline
            5 & \begin{tabular}{@{}c@{}} \cancel{(S, 0)} \cancel{(S, A, 3)} \cancel{(S, A, C, 5)} \cancel{(S, A, B, 7)} (S, A, C, 7) (S, A, B, D, 8) (S, G, 16) \end{tabular} \\
            \hline
            6 & \begin{tabular}{@{}c@{}} \cancel{(S, 0)} \cancel{(S, A, 3)} \cancel{(S, A, C, 5)} \cancel{(S, A, B, 7)} \cancel{(S, A, C, 7)} (S, A, B, D, 8) (S, A, C, E, 9) (S, G, 16) \end{tabular} \\
            \hline
            7 & \begin{tabular}{@{}c@{}} \cancel{(S, 0)} \cancel{(S, A, 3)} \cancel{(S, A, C, 5)} \cancel{(S, A, B, 7)} \cancel{(S, A, C, 7)} \cancel{(S, A, B, D, 8)} (S, A, C, E, 9) \\ (S, A, B, D, E, 10) (S, G, 16) \end{tabular} \\
            \hline
            8 & \begin{tabular}{@{}c@{}} \cancel{(S, 0)} \cancel{(S, A, 3)} \cancel{(S, A, C, 5)} \cancel{(S, A, B, 7)} \cancel{(S, A, C, 7)} \cancel{(S, A, B, D, 8)} \cancel{(S, A, C, E, 9)} \\ (S, A, B, D, E, 10) (S, A, C, E, G, 10) (S, G, 16) \end{tabular} \\
            \hline
            9 & \begin{tabular}{@{}c@{}} \cancel{(S, 0)} \cancel{(S, A, 3)} \cancel{(S, A, C, 5)} \cancel{(S, A, B, 7)} \cancel{(S, A, C, 7)} \cancel{(S, A, B, D, 8)} \cancel{(S, A, C, E, 9)} \\ \cancel{(S, A, B, D, E, 10)} (S, A, C, E, G, 10) (S, A, B, D, E, G, 11) (S, G, 16) \end{tabular} \\
            \hline
            10 & \begin{tabular}{@{}c@{}} \cancel{(S, 0)} \cancel{(S, A, 3)} \cancel{(S, A, C, 5)} \cancel{(S, A, B, 7)} \cancel{(S, A, C, 7)} \cancel{(S, A, B, D, 8)} \cancel{(S, A, C, E, 9)} \\ \cancel{(S, A, B, D, E, 10)} \cancel{(S, A, C, E, G, 10)} (S, A, B, D, E, G, 11) (S, G, 16) \end{tabular} \\
            \hline
        \end{tabular}
    \end{center}
    \end{latin}
    \item یکی از توابعی که می‌توانیم انتخاب کنیم که مشکلی نداشته باشد همان کوتاهترین فاصله از نودی که داخل آن هستیم تا
    $G$
    است!
    پس جدول زیر را در نظر بگیرید برای تابع
    $h$:
    \begin{latin}
    \begin{center}
        \begin{tabular}{|c|c|}
            \hline
            State & h1\\
            \hline
            S & 10\\
            \hline
            A & 7\\
            \hline
            B & 5\\
            \hline
            C & 5\\
            \hline
            D & 4\\
            \hline
            E & 1\\
            \hline
            G & 0\\
            \hline
        \end{tabular}
    \end{center}
    \end{latin}
    در
    $A^*$
    برای هر حال داریم که هزینه‌ی رفتن به یک حالت دیگر داریم:
    $f(n) = g(n) + h(n)$.
    پس جدول
    \lr{fringe}
    به صورت زیر می‌شود:
    \begin{latin}
    \begin{center}
        \begin{tabular}{|l|l|}
            \hline
            Step & Fringe\\
            \hline
            1 & \begin{tabular}{@{}c@{}}(S, 10)\end{tabular}  \\
            \hline
            2 & \begin{tabular}{@{}c@{}}\cancel{(S, 10)} (S, A, 10) (S, G, 16)\end{tabular}  \\
            \hline
            3 & \begin{tabular}{@{}c@{}}\cancel{(S, 10)} \cancel{(S, A, 10)} (S, A, C, 10) (S, A, B, 12) (S, G, 16)\end{tabular}  \\
            \hline
            4 & \begin{tabular}{@{}c@{}}\cancel{(S, 10)} \cancel{(S, A, 10)} \cancel{(S, A, C, 10)} (S, A, C, E, 10) (S, A, B, 12) (S, G, 16)\end{tabular}  \\
            \hline
            5 & \begin{tabular}{@{}c@{}}\cancel{(S, 10)} \cancel{(S, A, 10)} \cancel{(S, A, C, 10)} \cancel{(S, A, C, E, 10)} (S, A, B, 12) (S, G, 16)\end{tabular}  \\
            \hline
            6 & \begin{tabular}{@{}c@{}}\cancel{(S, 10)} \cancel{(S, A, 10)} \cancel{(S, A, C, 10)} \cancel{(S, A, C, E, 10)} (S, A, C, E, G, 10) (S, A, B, 12) (S, G, 16)\end{tabular}  \\
            \hline
            7 & \begin{tabular}{@{}c@{}}\cancel{(S, 10)} \cancel{(S, A, 10)} \cancel{(S, A, C, 10)} \cancel{(S, A, C, E, 10)} \cancel{(S, A, C, E, G, 10)} (S, A, B, 12) (S, G, 16)\end{tabular}  \\
            \hline
        \end{tabular}
    \end{center}
    \end{latin}
    \item $h1$
    قابل قبول است ولی $h2$ نیست. دلیل این موضوع این است که در حالت
    $B$
    مقدار
    $h$ بیشتر از مقدار واقعی کوتاهترین مسیر است.
    به عبارت دیگر بدبینانه است.
\end{enumerate}








